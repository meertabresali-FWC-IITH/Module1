\documentclass[conference]{IEEEtran}
\IEEEoverridecommandlockouts

% The preceding line is only needed to identify funding in the first footnote. If that is unneeded, please comment it out.
\usepackage{cite}
\usepackage{amsmath,amssymb,amsfonts}
\usepackage{graphicx}
\usepackage{textcomp}
\usepackage{xcolor}
\def\BibTeX{{\rm B\kern-.05em{\sc i\kern-.025em b}\kern-.08em
    T\kern-.1667em\lower.7ex\hbox{E}\kern-.125emX}}


\begin{document}
\title{
{Distributive Law of Boolean Algebra\\
AVR Assembly}\\

\thanks{Meer Tabres Ali as an intern with FWC IIT Hyderabad. *The author is with the Department of Electrical Engineering, Indian Institute of Technology, Hyderabad 502285 India e-mail: gadepall@iith.ac.in. All content in this manual is released under GNU GPL. Free and open source.}
}
\author{Meer Tabres Ali and G V V Sharma}
\maketitle

\section{ABSTRACT}
\begin{flushleft}
Distributive law of Boolean Algebra is expressed by the following expression.\\
X.(Y+Z) = X.Y + X.Z \\
In this program, it has been verified that both LHS and RHS parts of above distributive law are equal with the random inputs (X, Y, Z).\\
\end{flushleft}

\subsection{Truth Table for Distributive Law}
Truth Table for Distributive Law: X.(Y+Z) = X.Y + X.Z 

\begin{table}[htbp]
    \centering
\begin{tabular}{ | c | c | c | c | c | c | c | c | } \hline
\textbf{X} & \textbf{Y} & \textbf{Z} & \textbf{Y+Z} & \textbf{X(Y+Z)} & \textbf{XY} & \textbf{XZ} & \textbf{XY+XZ} \\\hline
0 & 0 & 0 & 0 & 0 & 0 & 0 & 0 \\
0 & 0 & 1 & 0 & 0 & 0 & 0 & 0 \\
0 & 1 & 0 & 0 & 0 & 0 & 0 & 0 \\
0 & 1 & 1 & 0 & 0 & 0 & 0 & 0 \\
1 & 0 & 0 & 0 & 0 & 0 & 0 & 0 \\
1 & 0 & 1 & 1 & 1 & 0 & 1 & 1 \\
1 & 1 & 0 & 1 & 1 & 1 & 0 & 1 \\
1 & 1 & 1 & 1 & 1 & 1 & 1 & 1 \\ \hline
\end{tabular}
\vspace{0.4cm}
\caption{\label{tab:widgets}Truth Table for Distributive Law1}
\end{table}


\section{COMPONENTS}
Required components list has been given in Table III
\begin{table}[h]
\centering
\begin{tabular}{| c | c | c |} \hline
\textbf{Components} & \textbf{Value} & \textbf{Quantity} \\\hline
Resistors & 220 ohm & 2 \\
LEDs &  & 2 \\
Arduino & UNO & 1 \\
Jumper Wires &  & 20 \\
Breadboard & & 1 \\ 
\hline
\end{tabular}
\vspace{0.4cm}
\caption{\label{tab:widgets}Components}
\end{table}

\section{HARDWARE}
\begin{flushleft}
For Distributive Law: X.(Y+Z) = X.Y + X.Z \\
Make the connections between Arduino and LEDs as per the Table IV. Here LHS=X.(Y+Z) and RHS=X.Y+X.Z\\

\begin{table}[h]
\begin{tabular}{|c | c | c | c | c | c |} \hline
 & \textbf{INPUT} & \textbf{INPUT} & \textbf{INPUT} & \textbf{OUTPUT} & \textbf{OUTPUT} \\\hline
 & X & Y & Z & X(Y+Z) & XY+XZ \\ \hline
\textbf{Arduino}& 11 & 12 & 13 & 8 & 2 \\ \hline
\textbf{LEDs} &  &  &  & LED1 & LED2 \\ \hline
\end{tabular}
\vspace{0.4cm}
\caption{\label{tab:widgets}Pin connections}
\end{table}
\end{flushleft}
\vspace{4cm}

\section{SOFTWARE}
\centering
1. Download the codes given in the link below and execute them.\\

\begin{table}[h]
\centering
\begin{tabular}{| c |} \hline
 \rule{0pt}{20pt} https://github.com/meertabresali-FWC-IITH/project/Assignment2/hello.asm\\\hline
\end{tabular}
\end{table}
\begin{flushleft}
2. Apply the inputs X, Y, and Z (either HIGH or LOW) to the Digital Pin no.s 11, 12 and 13 of Arduino as per the Truth table.\\
\vspace{0.4cm}
3. Output is taken from pin no.s 8 and 2. Pin 8 is assigned for LHS = X(Y+Z) and Pin 2 is assigned for RHS =(XY+XZ). 
\end{flushleft}

\section{FINDINGS}
\begin{flushleft}
For Distributive Law: X.(Y+Z) = X.Y + X.Z \\
\vspace{0.3cm}
After the execution of codes, it is verified that, for random input variables (X, Y and Z), the output pins (Pin8 for LHS=X(Y+Z) and Pin2 for RHS=XY+XZ) of Arduino are at the same level (i.e. both output pins are at either HIGH or LOW simultaneously), and it causes both LEDs (LED1 and LED2) either to glow or off.\\

\end{flushleft}
\section{CONCLUSION}
\begin{flushleft}
1. Distributive law is expressed by \\
X(Y+Z)=XY+XZ with LHS = X(Y+Z), RHS = XY+XZ, and \\
2. Codes are written for Distributive law and are executed.\\
3. Result has been shown by LEDs (i.e. LED1 and LED2). \\
4. LED1 is assigned for LHS of the Boolean expression of Distributive Law. \\
5. LED2 is assigned for RHS of the Boolean expression of Distributive Law. \\
6. For random digital inputs X, Y and Z as per Truth tables (at Arduino digital pins 11, 12 and 13), it has been noticed that, the output pins (8 and 2) of Arduino are at the same level.
\end{flushleft}
\end{document}